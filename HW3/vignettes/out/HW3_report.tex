% Options for packages loaded elsewhere
\PassOptionsToPackage{unicode}{hyperref}
\PassOptionsToPackage{hyphens}{url}
\PassOptionsToPackage{dvipsnames,svgnames*,x11names*}{xcolor}
%
\documentclass[
  11pt,
  a4paper,
]{scrartcl}
\usepackage{amsmath,amssymb}
\usepackage{lmodern}
\usepackage{ifxetex,ifluatex}
\ifnum 0\ifxetex 1\fi\ifluatex 1\fi=0 % if pdftex
  \usepackage[T1]{fontenc}
  \usepackage[utf8]{inputenc}
  \usepackage{textcomp} % provide euro and other symbols
\else % if luatex or xetex
  \usepackage{unicode-math}
  \defaultfontfeatures{Scale=MatchLowercase}
  \defaultfontfeatures[\rmfamily]{Ligatures=TeX,Scale=1}
\fi
% Use upquote if available, for straight quotes in verbatim environments
\IfFileExists{upquote.sty}{\usepackage{upquote}}{}
\IfFileExists{microtype.sty}{% use microtype if available
  \usepackage[]{microtype}
  \UseMicrotypeSet[protrusion]{basicmath} % disable protrusion for tt fonts
}{}
\makeatletter
\@ifundefined{KOMAClassName}{% if non-KOMA class
  \IfFileExists{parskip.sty}{%
    \usepackage{parskip}
  }{% else
    \setlength{\parindent}{0pt}
    \setlength{\parskip}{6pt plus 2pt minus 1pt}}
}{% if KOMA class
  \KOMAoptions{parskip=half}}
\makeatother
\usepackage{xcolor}
\IfFileExists{xurl.sty}{\usepackage{xurl}}{} % add URL line breaks if available
\IfFileExists{bookmark.sty}{\usepackage{bookmark}}{\usepackage{hyperref}}
\hypersetup{
  pdftitle={Homework 3 Report},
  pdfauthor={Ian Effendi \textbackslash{} iae2784@rit.edu},
  colorlinks=true,
  linkcolor=Maroon,
  filecolor=Maroon,
  citecolor=Blue,
  urlcolor=violet,
  pdfcreator={LaTeX via pandoc}}
\urlstyle{same} % disable monospaced font for URLs
\usepackage[margin=1in,heightrounded]{geometry}
\usepackage{color}
\usepackage{fancyvrb}
\newcommand{\VerbBar}{|}
\newcommand{\VERB}{\Verb[commandchars=\\\{\}]}
\DefineVerbatimEnvironment{Highlighting}{Verbatim}{commandchars=\\\{\}}
% Add ',fontsize=\small' for more characters per line
\usepackage{framed}
\definecolor{shadecolor}{RGB}{248,248,248}
\newenvironment{Shaded}{\begin{snugshade}}{\end{snugshade}}
\newcommand{\AlertTok}[1]{\textcolor[rgb]{0.94,0.16,0.16}{#1}}
\newcommand{\AnnotationTok}[1]{\textcolor[rgb]{0.56,0.35,0.01}{\textbf{\textit{#1}}}}
\newcommand{\AttributeTok}[1]{\textcolor[rgb]{0.77,0.63,0.00}{#1}}
\newcommand{\BaseNTok}[1]{\textcolor[rgb]{0.00,0.00,0.81}{#1}}
\newcommand{\BuiltInTok}[1]{#1}
\newcommand{\CharTok}[1]{\textcolor[rgb]{0.31,0.60,0.02}{#1}}
\newcommand{\CommentTok}[1]{\textcolor[rgb]{0.56,0.35,0.01}{\textit{#1}}}
\newcommand{\CommentVarTok}[1]{\textcolor[rgb]{0.56,0.35,0.01}{\textbf{\textit{#1}}}}
\newcommand{\ConstantTok}[1]{\textcolor[rgb]{0.00,0.00,0.00}{#1}}
\newcommand{\ControlFlowTok}[1]{\textcolor[rgb]{0.13,0.29,0.53}{\textbf{#1}}}
\newcommand{\DataTypeTok}[1]{\textcolor[rgb]{0.13,0.29,0.53}{#1}}
\newcommand{\DecValTok}[1]{\textcolor[rgb]{0.00,0.00,0.81}{#1}}
\newcommand{\DocumentationTok}[1]{\textcolor[rgb]{0.56,0.35,0.01}{\textbf{\textit{#1}}}}
\newcommand{\ErrorTok}[1]{\textcolor[rgb]{0.64,0.00,0.00}{\textbf{#1}}}
\newcommand{\ExtensionTok}[1]{#1}
\newcommand{\FloatTok}[1]{\textcolor[rgb]{0.00,0.00,0.81}{#1}}
\newcommand{\FunctionTok}[1]{\textcolor[rgb]{0.00,0.00,0.00}{#1}}
\newcommand{\ImportTok}[1]{#1}
\newcommand{\InformationTok}[1]{\textcolor[rgb]{0.56,0.35,0.01}{\textbf{\textit{#1}}}}
\newcommand{\KeywordTok}[1]{\textcolor[rgb]{0.13,0.29,0.53}{\textbf{#1}}}
\newcommand{\NormalTok}[1]{#1}
\newcommand{\OperatorTok}[1]{\textcolor[rgb]{0.81,0.36,0.00}{\textbf{#1}}}
\newcommand{\OtherTok}[1]{\textcolor[rgb]{0.56,0.35,0.01}{#1}}
\newcommand{\PreprocessorTok}[1]{\textcolor[rgb]{0.56,0.35,0.01}{\textit{#1}}}
\newcommand{\RegionMarkerTok}[1]{#1}
\newcommand{\SpecialCharTok}[1]{\textcolor[rgb]{0.00,0.00,0.00}{#1}}
\newcommand{\SpecialStringTok}[1]{\textcolor[rgb]{0.31,0.60,0.02}{#1}}
\newcommand{\StringTok}[1]{\textcolor[rgb]{0.31,0.60,0.02}{#1}}
\newcommand{\VariableTok}[1]{\textcolor[rgb]{0.00,0.00,0.00}{#1}}
\newcommand{\VerbatimStringTok}[1]{\textcolor[rgb]{0.31,0.60,0.02}{#1}}
\newcommand{\WarningTok}[1]{\textcolor[rgb]{0.56,0.35,0.01}{\textbf{\textit{#1}}}}
\usepackage{graphicx}
\makeatletter
\def\maxwidth{\ifdim\Gin@nat@width>\linewidth\linewidth\else\Gin@nat@width\fi}
\def\maxheight{\ifdim\Gin@nat@height>\textheight\textheight\else\Gin@nat@height\fi}
\makeatother
% Scale images if necessary, so that they will not overflow the page
% margins by default, and it is still possible to overwrite the defaults
% using explicit options in \includegraphics[width, height, ...]{}
\setkeys{Gin}{width=\maxwidth,height=\maxheight,keepaspectratio}
% Set default figure placement to htbp
\makeatletter
\def\fps@figure{htbp}
\makeatother
\setlength{\emergencystretch}{3em} % prevent overfull lines
\providecommand{\tightlist}{%
  \setlength{\itemsep}{0pt}\setlength{\parskip}{0pt}}
\setcounter{secnumdepth}{-\maxdimen} % remove section numbering
\ifluatex
  \usepackage{selnolig}  % disable illegal ligatures
\fi

\title{Homework 3 Report}
\usepackage{etoolbox}
\makeatletter
\providecommand{\subtitle}[1]{% add subtitle to \maketitle
  \apptocmd{\@title}{\par {\large #1 \par}}{}{}
}
\makeatother
\subtitle{Classification of liver malfunction severity (LDA)}
\author{Ian Effendi \textbackslash{}
\href{mailto:iae2784@rit.edu}{\nolinkurl{iae2784@rit.edu}}}
\date{October 11, 2021}

\begin{document}
\maketitle

{
\hypersetup{linkcolor=blue}
\setcounter{tocdepth}{3}
\tableofcontents
}
\hypertarget{certification}{%
\subsection{Certification}\label{certification}}

\begin{quote}
I certify that I indeed finished reading Ch. 4 from \emph{An
Introduction to Statistical Learning}, by James Gareth, Daniela Witten,
Trevor Hastie, Robert Tibshirani.
\end{quote}

\hypertarget{overview}{%
\subsection{Overview}\label{overview}}

\begin{Shaded}
\begin{Highlighting}[]
\CommentTok{\# Comments that begin with \textquotesingle{}\#\# {-}{-}{-}{-}\textquotesingle{} can be used}
\CommentTok{\# to separate content. R Studio provides them as }
\CommentTok{\# little bookmarks that you can use.}

\DocumentationTok{\#\# {-}{-}{-}{-} Hidden Variables {-}{-}{-}{-}}
\NormalTok{.VERBOSE }\OtherTok{=} \ConstantTok{TRUE}
\NormalTok{.FILEPATHS }\OtherTok{=} \FunctionTok{list}\NormalTok{(}
  \AttributeTok{utils      =} \FunctionTok{list}\NormalTok{(}
      \AttributeTok{.            =} \StringTok{"R/utils.R"}\NormalTok{,}
      \AttributeTok{dependencies =} \StringTok{"R/utils/dependencies.R"}\NormalTok{,}
      \AttributeTok{paths        =} \StringTok{"R/utils/paths.R"}\NormalTok{,}
      \AttributeTok{printf       =} \StringTok{"R/utils/printf.R"}
\NormalTok{  ),}
  \AttributeTok{packages   =} \StringTok{"R/packages.R"}\NormalTok{,}
  \AttributeTok{build      =} \FunctionTok{list}\NormalTok{(}
      \AttributeTok{.            =} \StringTok{"R/build.R"}\NormalTok{,}
      \AttributeTok{dataset      =} \StringTok{"R/build/dataset.R"}\NormalTok{,}
      \AttributeTok{report       =} \StringTok{"R/build/report.R"}
\NormalTok{  ),}
  \AttributeTok{analysis   =} \FunctionTok{list}\NormalTok{(}
      \AttributeTok{.            =} \StringTok{"R/analysis.R"}\NormalTok{,}
      \AttributeTok{setup        =} \StringTok{"R/analysis/setup.R"}\NormalTok{,}
      \AttributeTok{eda          =} \StringTok{"R/analysis/eda.R"}\NormalTok{,}
      \AttributeTok{model        =} \StringTok{"R/analysis/model.R"}\NormalTok{,}
      \AttributeTok{metrics      =} \StringTok{"R/analysis/metrics.R"}\NormalTok{,}
      \AttributeTok{validation   =} \StringTok{"R/analysis/validation.R"}
\NormalTok{  )}
\NormalTok{)}

\DocumentationTok{\#\# {-}{-}{-}{-} Default Settings (Knit) {-}{-}{-}{-}}
\NormalTok{knitr}\SpecialCharTok{::}\NormalTok{opts\_knit}\SpecialCharTok{$}\FunctionTok{set}\NormalTok{( }
  \AttributeTok{root.dir =}\NormalTok{ here}\SpecialCharTok{::}\FunctionTok{here}\NormalTok{()}
\NormalTok{)}

\DocumentationTok{\#\# {-}{-}{-}{-} Default Settings (Chunk) {-}{-}{-}{-}}
\NormalTok{knitr}\SpecialCharTok{::}\NormalTok{opts\_chunk}\SpecialCharTok{$}\FunctionTok{set}\NormalTok{(}
  
  \DocumentationTok{\#\# {-}{-}{-}{-} Options (Execution) {-}{-}{-}{-}}
  \AttributeTok{eval        =} \ConstantTok{TRUE}\NormalTok{,   }\CommentTok{\# By default, turn off chunk evaluation.}
  \AttributeTok{cache       =} \ConstantTok{FALSE}\NormalTok{,  }\CommentTok{\# By default, cache all chunk results.}
  \AttributeTok{error       =} \ConstantTok{FALSE}\NormalTok{,   }\CommentTok{\# Stop R when an error is raised.}
  
  \DocumentationTok{\#\# {-}{-}{-}{-} Options (Figure) {-}{-}{-}{-}}
  \AttributeTok{fig.path    =} \StringTok{\textquotesingle{}figure/\textquotesingle{}}\NormalTok{,}
  \AttributeTok{fig.keep    =} \StringTok{\textquotesingle{}all\textquotesingle{}}\NormalTok{,}
  \AttributeTok{fig.align   =} \StringTok{\textquotesingle{}center\textquotesingle{}}\NormalTok{,}
  \AttributeTok{fig.width   =} \DecValTok{8}\NormalTok{,}
  \AttributeTok{fig.height  =} \DecValTok{6}\NormalTok{,}
  \AttributeTok{dpi         =} \DecValTok{600}\NormalTok{,}
  
  \DocumentationTok{\#\# {-}{-}{-}{-} Options (Formatting) {-}{-}{-}{-}}
  \AttributeTok{results     =} \StringTok{\textquotesingle{}hold\textquotesingle{}}\NormalTok{,}
  \AttributeTok{collapse    =} \ConstantTok{FALSE}\NormalTok{,}
  \AttributeTok{strip.white =} \ConstantTok{FALSE}\NormalTok{,  }\CommentTok{\# Remove blank lines at beginning and end of the source code block output.}
  \AttributeTok{tidy        =} \ConstantTok{FALSE}\NormalTok{,}
  \AttributeTok{size        =} \StringTok{\textquotesingle{}tiny\textquotesingle{}}\NormalTok{,}
  \AttributeTok{R.options   =} \FunctionTok{list}\NormalTok{(}\AttributeTok{width =} \DecValTok{60}\NormalTok{),}

  \DocumentationTok{\#\# {-}{-}{-}{-} Options (Display) {-}{-}{-}{-}}
  \AttributeTok{include     =} \ConstantTok{TRUE}\NormalTok{,}
  \AttributeTok{echo        =} \ConstantTok{TRUE}\NormalTok{,   }
  \AttributeTok{message     =} \ConstantTok{TRUE}\NormalTok{,}
  \AttributeTok{warning     =} \ConstantTok{TRUE}
\NormalTok{)}

\DocumentationTok{\#\# {-}{-}{-}{-} Utility Functions {-}{-}{-}{-}}

\CommentTok{\#\textquotesingle{} Use relative path to import all}
\CommentTok{\#\textquotesingle{} chunks from an external script,}
\CommentTok{\#\textquotesingle{} without executing the code inside}
\CommentTok{\#\textquotesingle{} of the document.}
\CommentTok{\#\textquotesingle{}}
\CommentTok{\#\textquotesingle{} @param path Relative path to file.}
\CommentTok{\#\textquotesingle{} @param ... Additional parameters passed to the \textasciigrave{}read\_chunk\textasciigrave{} call.}
\CommentTok{\#\textquotesingle{} @param roxygen\_comments FALSE; should comments be stripped out from external code?}
\CommentTok{\#\textquotesingle{} @param verbose Print messages to stderr.}
\NormalTok{import.chunks }\OtherTok{\textless{}{-}} \ControlFlowTok{function}\NormalTok{(path, ..., }\AttributeTok{verbose =}\NormalTok{ .VERBOSE) \{}
  \ControlFlowTok{if}\NormalTok{ (verbose) \{ }\FunctionTok{message}\NormalTok{(}\FunctionTok{sprintf}\NormalTok{(}\StringTok{"Importing chunks from external script: \textquotesingle{}\%s\textquotesingle{}..."}\NormalTok{, }\FunctionTok{basename}\NormalTok{(path))) \}}
\NormalTok{  file }\OtherTok{=}\NormalTok{ here}\SpecialCharTok{::}\FunctionTok{here}\NormalTok{(path)}
\NormalTok{  knitr}\SpecialCharTok{::}\FunctionTok{read\_chunk}\NormalTok{(file, ...)}
\NormalTok{\}}

\CommentTok{\#\textquotesingle{} Use relative path to source an}
\CommentTok{\#\textquotesingle{} external script in its entirety.}
\CommentTok{\#\textquotesingle{}}
\CommentTok{\#\textquotesingle{} @param path Relative path to file.}
\CommentTok{\#\textquotesingle{} @param ... Additional parameters passed to the \textasciigrave{}source\textasciigrave{} call.}
\CommentTok{\#\textquotesingle{} @param verbose Print messages to stderr.}
\NormalTok{import.source }\OtherTok{\textless{}{-}} \ControlFlowTok{function}\NormalTok{(path, ..., }\AttributeTok{verbose =}\NormalTok{ .VERBOSE) \{}
  \ControlFlowTok{if}\NormalTok{ (verbose) \{ }\FunctionTok{message}\NormalTok{(}\FunctionTok{sprintf}\NormalTok{(}\StringTok{"Sourcing external script: \textquotesingle{}\%s\textquotesingle{}..."}\NormalTok{, }\FunctionTok{basename}\NormalTok{(path))) \}}
  \FunctionTok{source}\NormalTok{(here}\SpecialCharTok{::}\FunctionTok{here}\NormalTok{(path), ..., }\AttributeTok{echo =}\NormalTok{ .VERBOSE)}
\NormalTok{\}}
\end{Highlighting}
\end{Shaded}

\begin{Shaded}
\begin{Highlighting}[]

\DocumentationTok{\#\# {-}{-}{-}{-} Import (R/utils.R) {-}{-}{-}{-}}
\FunctionTok{import.chunks}\NormalTok{(.FILEPATHS}\SpecialCharTok{$}\NormalTok{utils}\SpecialCharTok{$}\NormalTok{.)}
\end{Highlighting}
\end{Shaded}

\begin{verbatim}
## Importing chunks from external script: 'utils.R'...
\end{verbatim}

\begin{Shaded}
\begin{Highlighting}[]

\DocumentationTok{\#\# {-}{-}{-}{-} Import (R/packages.R) {-}{-}{-}{-}}
\FunctionTok{import.chunks}\NormalTok{(.FILEPATHS}\SpecialCharTok{$}\NormalTok{packages)}
\end{Highlighting}
\end{Shaded}

\begin{verbatim}
## Importing chunks from external script: 'packages.R'...
\end{verbatim}

\begin{Shaded}
\begin{Highlighting}[]

\DocumentationTok{\#\# {-}{-}{-}{-} Import (R/build/*) {-}{-}{-}{-}}
\FunctionTok{import.chunks}\NormalTok{(.FILEPATHS}\SpecialCharTok{$}\NormalTok{build}\SpecialCharTok{$}\NormalTok{dataset)}
\end{Highlighting}
\end{Shaded}

\begin{verbatim}
## Importing chunks from external script: 'dataset.R'...
\end{verbatim}

\begin{Shaded}
\begin{Highlighting}[]
\FunctionTok{import.chunks}\NormalTok{(.FILEPATHS}\SpecialCharTok{$}\NormalTok{build}\SpecialCharTok{$}\NormalTok{report)}
\end{Highlighting}
\end{Shaded}

\begin{verbatim}
## Importing chunks from external script: 'report.R'...
\end{verbatim}

\begin{Shaded}
\begin{Highlighting}[]

\DocumentationTok{\#\# {-}{-}{-}{-} Import (R/analysis/*) {-}{-}{-}{-}}
\end{Highlighting}
\end{Shaded}

\begin{Shaded}
\begin{Highlighting}[]

\DocumentationTok{\#\# {-}{-}{-}{-} Source (R/utils.R) {-}{-}{-}{-}}
\FunctionTok{source.utils}\NormalTok{(}\AttributeTok{local =} \ConstantTok{FALSE}\NormalTok{, }\AttributeTok{force =} \ConstantTok{TRUE}\NormalTok{, }\AttributeTok{verbose =}\NormalTok{ .VERBOSE)}
\end{Highlighting}
\end{Shaded}

\begin{verbatim}
## Loading 3 utilit(y/ies)...
\end{verbatim}

\begin{verbatim}
## Loading utility: 'printf'
\end{verbatim}

\begin{verbatim}
## Loading utility: 'paths'
\end{verbatim}

\begin{verbatim}
## Loading utility: 'dependencies'
\end{verbatim}

\begin{Shaded}
\begin{Highlighting}[]

\DocumentationTok{\#\# {-}{-}{-}{-} Source (R/packages.R) {-}{-}{-}{-}}
\FunctionTok{install.deps}\NormalTok{(}\AttributeTok{file =} \StringTok{"requirements.txt"}\NormalTok{)}
\end{Highlighting}
\end{Shaded}

\begin{verbatim}
## * The project is already synchronized with the lockfile.
\end{verbatim}

In this assignment we will:

\begin{itemize}
\tightlist
\item
  Perform exploratory data analysis (\emph{EDA}) on the dataset.
\item
  Fit and analyze a linear discriminant analysis (\emph{LDA}) model on
  the dataset.
\item
  Perform multiple cross-validation tasks at different \(k\)-fold values
  (\(k=3,\,k=10\)).
\end{itemize}

\hypertarget{elt}{%
\subsection{ELT}\label{elt}}

Much of the \emph{extract}, \emph{load}, and \emph{transform}
(\emph{ELT}) process from the previous report has been revised for this
assignment. Notably, a \texttt{make.dataset()} function streamlines the
process of parsing the source \texttt{Liver.txt} file into a compatible
\texttt{data.frame}.

\newpage

\hypertarget{session-information}{%
\subsection{Session Information}\label{session-information}}

\emph{This document was generated from an
\href{http://rmarkdown.rstudio.com}{R Markdown} Notebook (See the
\texttt{vignettes/HW3\_report.Rmd} in the project's sub-directory). The
setup chunk for this document sets the root directory to the project
root directory using the \texttt{here} package; all file paths are
relative to the project root.}

\begin{verbatim}
## R version 4.1.1 (2021-08-10)
## Platform: x86_64-w64-mingw32/x64 (64-bit)
## Running under: Windows 10 x64 (build 19042)
## 
## Matrix products: default
## 
## locale:
## [1] LC_COLLATE=English_United States.1252 
## [2] LC_CTYPE=English_United States.1252   
## [3] LC_MONETARY=English_United States.1252
## [4] LC_NUMERIC=C                          
## [5] LC_TIME=English_United States.1252    
## 
## attached base packages:
## [1] stats     graphics  grDevices datasets  utils    
## [6] methods   base     
## 
## other attached packages:
##  [1] dplyr_1.0.7    tidyr_1.1.4    forcats_0.5.1 
##  [4] ggplot2_3.3.5  foreach_1.5.1  magrittr_2.0.1
##  [7] mime_0.12      markdown_1.1   rmarkdown_2.11
## [10] knitr_1.36    
## 
## loaded via a namespace (and not attached):
##  [1] compiler_4.1.1   pillar_1.6.3     jquerylib_0.1.4 
##  [4] iterators_1.0.13 tools_4.1.1      bit_4.0.4       
##  [7] digest_0.6.28    corrplot_0.90    jsonlite_1.7.2  
## [10] evaluate_0.14    lifecycle_1.0.1  tibble_3.1.5    
## [13] gtable_0.3.0     pkgconfig_2.0.3  rlang_0.4.11    
## [16] parallel_4.1.1   yaml_2.2.1       xfun_0.26       
## [19] fastmap_1.1.0    withr_2.4.2      stringr_1.4.0   
## [22] hms_1.1.1        generics_0.1.0   vctrs_0.3.8     
## [25] bit64_4.0.5      tidyselect_1.1.1 rprojroot_2.0.2 
## [28] grid_4.1.1       glue_1.4.2       here_1.0.1      
## [31] R6_2.5.1         fansi_0.5.0      vroom_1.5.5     
## [34] tzdb_0.1.2       readr_2.0.2      purrr_0.3.4     
## [37] scales_1.1.1     codetools_0.2-18 htmltools_0.5.2 
## [40] ellipsis_0.3.2   colorspace_2.0-2 renv_0.14.0     
## [43] utf8_1.2.2       stringi_1.7.5    munsell_0.5.0   
## [46] crayon_1.4.1
\end{verbatim}

\end{document}
